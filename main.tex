\documentclass[12pt]{article}

%DEFINIZIONE DEI PACCHETTI GENERICI
\usepackage[a4paper,top=3.5cm,bottom=2.5cm,left=3cm,right=3.5cm]{geometry}
\usepackage{times}
\usepackage{titlesec}
%\usepackage{lipsum}
\usepackage{titletoc}

%DEFINIZIONE DEI PACCHETTI MATEMATICI
\usepackage{amsmath}
\usepackage{amsfonts}
\usepackage{amssymb}
\usepackage{amsthm}

%STILE DELLE PROPOSIZIONI
\theoremstyle{plain}

% Intestazioni in ingelse
\newtheorem{thm}{Theorem}[section]
\newtheorem{lem}[thm]{Lemma}
\newtheorem{prop}[thm]{Proposition}
\newtheorem{cor}[thm]{Corollary}
\newtheorem{defn}[thm]{Definition}
\newtheorem{rmk}[thm]{Remark}
\newtheorem{ex}[thm]{Example}
\newtheorem{prob}[thm]{Problem}
\newtheorem*{quest*}{Question}
\newtheorem*{notat*}{Notation}

% Intestazioni in italiano
%\newtheorem{thm}{Teorema}[section]
%\newtheorem{lem}[thm]{Lemma}
%\newtheorem{prop}[thm]{Proposizione}
%\newtheorem{cor}[thm]{Corollario}
%\newtheorem{defn}[thm]{Definizione}
%\newtheorem{rmk}[thm]{Osservazione}
%\newtheorem{ex}[thm]{Esempio}
%\newtheorem{prob}[thm]{Problema}
%\newtheorem*{notat*}{Notazione}
%\newtheorem*{quest*}{Domanda}

%STILE DELLE SEZIONI
\titleformat{\section}
{\normalfont \scshape \centering}{\thesection.}{0,5em}{}
%\titlespacing*{\section}{0pt}{50pt}{0.5cm}

\titleformat{\subsection}
{\normalfont \bfseries }{\normalfont \thesubsection.}{0,5em}{}
%\titlespacing*{\subsection}{0pt}{50pt}{0.5cm}

\titlecontents{section}[0cm]{}
{\normalfont \thecontentslabel. \enspace}
{\hspace*{-5.3em}}
{ \hfill \normalfont \contentspage}

\titlecontents{subsection}[0cm]{}
{\normalfont \thecontentslabel. \enspace}
{\hspace*{-5.3em}}
{ \hfill \normalfont \contentspage}

%OPZIONI PER LA BIBLIOGRAFIA
\usepackage[
%backend=bibtex,
backend=biber,
style=alphabetic,
]{biblatex}
\addbibresource{bibliography.bib}
\AtNextBibliography{\small}

% imposto lo stile dell'abstract
\renewcommand{\abstractname}{\normalfont \scshape \centering Abstract}

%INIZIO DEL DOCUMENTO
\begin{document}

    % impostazione del titolo
	\begin{center}
	    \fontsize{12pt}{0pt} % dimensioni del titolo
        \textbf{NEURAL NETWORKS FOUNDATIONS} % titolo
	\end{center}

    %impostazione dell'indice
    {
    \fontsize{12pt}{0pt} % dimensioni delle voci
    \tableofcontents % print della tavola dei contenuti
    }

    \fontsize{12pt}{0pt} % dimensione del font e spaziatura tra le righe
    %INIZIA A SCRIVERE QUI

    \section{Vector spaces and linear 1-forms}
    In this chapter we define vector spaces and the linear functions between them.

    We begin by talking of groups.
    \begin{defn}
        Let $G \neq \emptyset$ a set, with a function, called operation
        \[ \cdot : G \times G \longrightarrow G.\]
        We say that $(G,+)$ is a \textbf{group} if
        \begin{enumerate}
            \item[(a)] there exists $e \in G$ such that $g\cdot e = g$, for all $g \in G$;
            \item[(b)] for all $g \in G$ the exists $h \in G$ such that $g \cdot h = e$;
            \item[(c)] the operation is associative, i.e. for all $g, h, t \in G$
            \[ g \cdot (h \cdot t) = (g \cdot h) \cdot t.\]
        \end{enumerate}
    \end{defn}

    \begin{rmk}
        In general the operation on $G$ is not commutative.
    \end{rmk}

    \begin{defn}
        Give a group $(G, \cdot)$, we say that $G$ is an $\textbf{abelian group}$ if the operation is commutative, i.e.
        \[ g_1 \cdot g_2 = g_2 \cdot g_1, \, \forall g_1, g_2 \in G.\]
    \end{defn}

    \begin{defn}
        Let $A \neq \emptyset$ a set. Consider two operation on $A$ 
        \[ + : A \times A \longrightarrow A\]
        \[ \cdot : A \times A \longrightarrow A,\]

        we define the triplets $(A, +, \cdot)$ to be a ring if:
        \begin{enumerate}
            \item[(1)] $(A, +)$ is an abelian group with identity element $0$;
            \item[(2)] $\cdot$ is an associative operation;
            \item[(3)] it holds the distributive property
            \[ a \cdot (b + c) = a \cdot b + a \cdot c\]
            and \[ (b +c ) \cdot a = b \cdot a + c \cdot a;\]
            \item[(4)] $a \cdot 0 = 0 \cdot a = 0.$
        \end{enumerate}
        We define $(A, +, \cdot)$ to be a unitary ring if there exits an alement $1 \in A$ such that
        \[ a \cdot 1 = 1 \cdot a = a,\, \forall a \in A.\]
        We define $(A, +, \cdot)$ to be a commutative ring if, for all $a,b \in A$ it holds
        \[ a \cdot b = b \cdot a.\]
    \end{defn}

    \begin{defn}
        We call a ring $A$ a \textbf{field} if its commutative and for every $a \in A$ there exists $b \in A$ such that $a\cdot b = 1$. We denote $b = a^{-1}$.
    \end{defn}

    \begin{defn}
        A \textbf{vector space over a field k} is an abelian group $(V,+)$ with an operation called \textbf{multiplication for a scalar} 
        \[ k \times V \longrightarrow V\] \[ (k,v) \longmapsto k\cdot v.\]
    \end{defn}

    \begin{defn}
        Consider two vector spaces $V, W$ and a map $f:V \longrightarrow W$. We call $f$ to be \textbf{k-linear} (or \textbf{linear}) if 
        \[ f(v+w) = f(v) + f(w)\]
        and 
        \[ f(\lambda \cdot v) = \lambda \cdot f(v).\]
    \end{defn}

    \begin{defn}
        Given a vector space $V$ we call \textbf{1-form} a linear function $f :V \longrightarrow k$.
        The set of all 1-forms on the vector space $V$ is called \textbf{dual vector space} and is denoted with $V^*$.
    \end{defn}
    
    \section{Neurons}
    In this chapter we introduce and study neurons. We are gonna start by the more basic one and proceed adding things until we reach a complete and full explanations of how neurons works and what they can do.

    During this chapter we are going to introduce also notions of linear algebra, affine geometry and calculus to give to neurons a mathematical reason.

    For coding in this first chapter we are going to use C and Python.
    

    \subsection{Basic neuron}
    

	%STAMPA DELLA BIBLIOGRAFIA
	\printbibliography[title=References and Bibliography]
	
\end{document}