\documentclass[12pt]{article}

%DEFINIZIONE DEI PACCHETTI GENERICI
\usepackage[a4paper,top=3.5cm,bottom=2.5cm,left=3cm,right=3.5cm]{geometry}
\usepackage{times}
\usepackage{titlesec}
%\usepackage{lipsum}
\usepackage{titletoc}

%DEFINIZIONE DEI PACCHETTI MATEMATICI
\usepackage{amsmath}
\usepackage{amsfonts}
\usepackage{amssymb}
\usepackage{amsthm}

%STILE DELLE PROPOSIZIONI
\theoremstyle{plain}

% Intestazioni in ingelse
%\newtheorem{thm}{Theorem}[section]
%\newtheorem{lem}[thm]{Lemma}
%\newtheorem{prop}[thm]{Proposition}
%\newtheorem{cor}[thm]{Corollary}
%\newtheorem{defn}[thm]{Definition}
%\newtheorem{rmk}[thm]{Remark}
%\newtheorem{ex}[thm]{Example}
%\newtheorem{prob}[thm]{Problem}
%\newtheorem*{notat*}{Notation}

% Intestazioni in italiano
\newtheorem{thm}{Teorema}[section]
\newtheorem{lem}[thm]{Lemma}
\newtheorem{prop}[thm]{Proposizione}
\newtheorem{cor}[thm]{Corollario}
\newtheorem{defn}[thm]{Definizione}
\newtheorem{rmk}[thm]{Osservazione}
\newtheorem{ex}[thm]{Esempio}
\newtheorem{prob}[thm]{Problema}
\newtheorem*{notat*}{Notazione}

%STILE DELLE SEZIONI
\titleformat{\section}
{\normalfont \scshape \centering}{\thesection.}{0,5em}{}
\titlespacing*{\section}{0pt}{50pt}{0.5cm}

\titleformat{\subsection}
{\normalfont \bfseries }{ \normalfont \thesubsection.}{0,5em}{}
\titlespacing*{\subsection}{0pt}{50pt}{0.5cm}

\titlecontents{section}[0cm]{}
{\normalfont \thecontentslabel. \enspace}
{\hspace*{-5.3em}}
{ \hfill \normalfont \contentspage}

\titlecontents{subsection}[0cm]{}
{\normalfont \thecontentslabel. \enspace}
{\hspace*{-5.3em}}
{ \hfill \normalfont \contentspage}

%OPZIONI PER LA BIBLIOGRAFIA
\usepackage[
backend=bibtex,
style=alphabetic,
]{biblatex}
\addbibresource{bibliography.bib}
\AtNextBibliography{\small}


%INIZIO DEL DOCUMENTO
\begin{document}
	
	\fontsize{15pt}{0pt}
	\begin{center}
		\textbf{Neural Networks Foundations}
	\end{center}

    \fontsize{12pt}{0pt}
	\tableofcontents

    %INIZIA A SCRIVERE QUI

    \section{Neurons}
    In this chapter we introduce and study neurons. We are gonna start by the more basic one and proceed adding things until we reach a complete and full explanations of how neurons works and what they can do.

    During this chapter we are going to introduce also notions of linear algebra, affine geometry and calculus to give to neurons a mathematical reason.

    For coding in this first chapter we are going to use C and Python.
    
    \vspace{-1.5cm}
    \subsection{Basic neuron}
    
	%STAMPA DELLA BIBLIOGRAFIA
	\printbibliography[title=References and Bibliography]
	
\end{document}
